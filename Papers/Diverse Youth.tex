\cvlinkevent{2018}{Engaging diverse Canadian youth in youth development programs: Program quality and community engagement}{DOI: \hyperlink{https://doi.org/10.1016/j.childyouth.2018.09.023}{https://doi.org/10.1016/j.childyouth.2018.09.023}}{Data Analyst}{Toronto \color{cvred}}{Youth development programs are key tools in promoting community engagement, which is a core feature of positive youth development. However, further research is needed on program quality and outcomes for diverse samples of youth. We examined program quality (positive features and youth-adult partnership) within youth programs, as predictors of three indicators of community engagement in a diverse youth sample (N = 321; Mean age = 16.2 years; SD = 3.0). Both positive program features and youth-adult partnership were positively related to youth civic participation, sociopolitical empowerment, and sense of community. Among our background variables, only LGBTQ status, perceived income, and age were related to community engagement. Positive associations between program quality and community engagement held across sample characteristics. Findings add to the limited research on youth development programs and youth's community engagement.}{the_students_commission_of_canada_logo.jpg}