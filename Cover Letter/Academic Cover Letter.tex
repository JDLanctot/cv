Dear Université Laval Search Committee,
\vspace{1.0em}

\hspace{1.5em} I am writing to apply for the tenure-track Assistant Professor position in theoretical physics at Université Laval. As a PhD candidate in Complex Systems at Toronto Metropolitan University with expertise in statistical mechanics, network theory, and computational physics; my research directly aligns with your department's focus areas of theoretical high-energy physics, astrophysics, and complex systems. My work applies statistical mechanics principles to understand universal phenomena across scales, from mitochondrial network dynamics to prediction market behavior, positioning me to contribute meaningfully to your theoretical physics program.
\vspace{1.0em}

\hspace{1.5em} My research program centers on identifying universal principles that govern complex systems through rigorous statistical mechanics and network theory approaches. With over \$45,000 in competitive funding (NSERC, OGS), I have developed mathematical frameworks that reveal scaling behaviors and phase transitions in diverse systems. My published work in Physical Review X Life demonstrates how network topology optimization emerges in biological systems—research that directly informs my approach to understanding critical phenomena and emergent structures in theoretical physics. This foundation in statistical mechanics, combined with computational expertise in Python, PyTorch, and Julia, enables me to tackle both analytical and numerical challenges in theoretical physics research.
\vspace{1.0em}

\hspace{1.5em} My interdisciplinary approach has proven essential for advancing theoretical understanding. My adversarial AI research, achieving exceptional effectiveness in attack mitigation in critical infrastructure networks, employs information-theoretic frameworks that connect to quantum field theory and many-body physics concepts. Similarly, my analysis of millions of mobile phone records during COVID-19 reveals collective behavior patterns that mirror phase transitions studied in statistical mechanics and cosmology. These connections between seemingly disparate fields exemplify the interdisciplinary collaboration that Université Laval encourages through its partnerships with CRM, CRAQ, CIMM, and the Centre de recherche en données massives.
\vspace{1.0em}

\hspace{1.5em} My research in prediction market universality has identified scaling laws that parallel critical phenomena in gauge theories, demonstrating how complex systems research can illuminate fundamental physics questions. Processing datasets exceeding 10 million data points has developed my expertise in computational approaches essential for modern theoretical physics. My record of 5+ peer-reviewed publications, presentations at the American Physical Society March Meeting, and successful grant acquisition demonstrates both research productivity and the communication skills necessary for an independent academic career. This computational and analytical foundation positions me to contribute to theoretical physics research that leverages large-scale simulations and data analysis.
\vspace{1.0em}

\hspace{1.5em} En tant que francophone, je considère cette opportunité comme un retour naturel à mes origines linguistiques dans un contexte académique d'excellence. Mon bilinguisme me permettra de contribuer efficacement à l'enseignement et à la recherche en français, tout en facilitant les collaborations internationales. J'apprécie particulièrement la possibilité de développer ma carrière dans l'environnement francophone de l'Université Laval, où je pourrai allier excellence scientifique et identité culturelle dans mes activités d'enseignement et de recherche.
\vspace{1.0em}

\hspace{1.5em} My teaching experience includes mentoring hundreds of undergraduate students in Python-based data analysis, statistical modeling, and scientific computation—skills increasingly essential in theoretical physics education. This experience, combined with my research background, prepares me to contribute to curriculum development and graduate student supervision in theoretical physics. I am particularly interested in collaborating with CRAQ on cosmological modeling projects, with CRM on mathematical physics applications, and with the Centre de recherche en données massives on computational approaches to theoretical problems. The January 2026 start date aligns well with my PhD completion timeline and would allow for seamless transition to establishing an independent research program at Université Laval.
\vspace{1.0em}

Sincerely,\\
\myName\\[1.0em]
% \href{\myEmailHref}{%
%     \infobubblesm{\faAt}{cvblue}{iconcolour}{\myEmail}{}
% }\\[0.1em]
% \href{\myGitHubHref}{%
%     \infobubblesm{\faGithub}{cvblue}{iconcolour}{\myGitHub}{}
% }\\[0.1em]
% \href{\myLinkedInHref}{%
%     \infobubblesm{\faLinkedin}{cvblue}{iconcolour}{\myLinkedIn}{}
% }\\[0.1em]

