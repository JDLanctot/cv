Dear Dr. Michael Kolios,
\vspace{1.0em}

\hspace{1.5em} I am writing to reflect on the process of revising my initial AI-focused cover letter into the Academic version that ultimately proved more effective for the Université Laval theoretical physics position. This revision process revealed fundamental insights about positioning interdisciplinary research within traditional physics departments.
\vspace{1.0em}

\hspace{1.5em} My initial draft emphasized artificial intelligence and machine learning capabilities, positioning these as primary qualifications for the theoretical physics role. Upon reflection, I recognized this approach fundamentally misunderstood what the search committee sought. I revised the letter to position my computational expertise as tools that serve theoretical physics inquiry rather than as ends in themselves. This shift required reframing my research narrative to demonstrate how complex systems methodologies illuminate fundamental physical principles.
\vspace{1.0em}

\hspace{1.5em} The revision process demanded careful attention to disciplinary language and conceptual frameworks. I replaced AI-centric terminology with precise theoretical physics concepts—critical phenomena, phase transitions, gauge theories—that demonstrate fluency in the field's intellectual traditions. My work on mitochondrial networks and prediction markets became evidence of how complex systems thinking advances understanding of universal physical principles, rather than showcasing computational technique for its own sake.
\vspace{1.0em}

\hspace{1.5em} Most significantly, I restructured the argument to address the core challenge facing interdisciplinary candidates: proving that cross-disciplinary perspective enhances rather than dilutes traditional expertise. The Academic letter positions my background as that of a theoretical physicist who leverages computational methods, not an AI researcher applying techniques to physics problems. This distinction matters profoundly to search committees evaluating candidates for theoretical physics positions.
\vspace{1.0em}

\hspace{1.5em} This revision process clarified that successful academic positioning requires understanding what drives a field intellectually, not merely demonstrating technical competencies. For theoretical physics, this means showing commitment to fundamental questions about universal principles and mathematical structures governing natural phenomena. The Academic letter succeeds by demonstrating that my interdisciplinary background serves these enduring physics concerns rather than replacing them with computational priorities.
\vspace{1.0em}

Sincerely,\\
\myName\\[1.0em]
